\documentclass[a4paper]{article}
\usepackage[T1]{fontenc}			% \chapter package
\usepackage[english]{babel}
\usepackage[english]{isodate}  		% date format
\usepackage{graphicx}				% manage images
\usepackage{amsfonts}
\usepackage{booktabs}				% high quality tables
\usepackage{amsmath}				% math package
\usepackage{amssymb}				% another math package (e.g. \nexists)
\usepackage{bm}                     % bold math symbols
\usepackage{mathtools}				% emphasize equations
\usepackage{stmaryrd} 				% '\llbracket' and '\rrbracket'
\usepackage{amsthm}					% better theorems
\usepackage{enumitem}				% manage list
\usepackage{pifont}					% nice itemize
\usepackage{cancel}					% cancel math equations
\usepackage{caption}				% custom caption
\usepackage[]{mdframed}				% box text
\usepackage{multirow}				% more lines in a table
\usepackage{textcomp, gensymb}		% degree symbol
\usepackage[x11names]{xcolor}		% RGB color
\usepackage[many]{tcolorbox}		% colorful box
\usepackage{multicol}				% more rows in a table (used for the lists)
\usepackage{listings}
\usepackage{url}
\usepackage{qrcode}
\usepackage{fontawesome5}
\usepackage{ragged2e}
\usepackage{cite}                   % references
\usepackage{imakeidx}               % index
\makeindex[program=makeindex, columns=2,
           title=Index, 
           intoc,
           options={-s index-style.ist}]


\definecolor{codegreen}{rgb}{0,0.6,0}
\definecolor{codegray}{rgb}{0.5,0.5,0.5}
\definecolor{codepurple}{rgb}{0.58,0,0.82}
\definecolor{backcolour}{rgb}{0.95,0.95,0.92}
\lstdefinestyle{mystyle}{
    backgroundcolor=\color{backcolour},   
    commentstyle=\color{codegreen},
    keywordstyle=\color{magenta},
    numberstyle=\tiny\color{codegray},
    stringstyle=\color{codepurple},
    basicstyle=\ttfamily\footnotesize,
    breakatwhitespace=false,         
    breaklines=true,                 
    captionpos=b,                    
    keepspaces=true,                 
    numbers=left,                    
    numbersep=5pt,                  
    showspaces=false,                
    showstringspaces=false,
    showtabs=false,                  
    tabsize=2
}
\lstset{
  language=R,
  basicstyle=\footnotesize\ttfamily,
  literate=~{$\sim$}2
}
\lstset{style=mystyle}


% draw a frame around given text
\newcommand{\framedtext}[1]{%
	\par%
	\noindent\fbox{%
		\parbox{\dimexpr\linewidth-2\fboxsep-2\fboxrule}{#1}%
	}%
}


% table of content links
\usepackage{xcolor}
\usepackage[linkcolor=black, citecolor=blue, urlcolor=cyan]{hyperref} % hypertexnames=false
\hypersetup{
	colorlinks=true
}


\newtheorem{theorem}{\textcolor{Red3}{\underline{Theorem}}}
\renewcommand{\qedsymbol}{QED}
\newcommand{\dquotes}[1]{``#1''}
\newcommand{\longline}{\noindent\rule{\textwidth}{0.4pt}}
\newcommand{\circledtext}[1]{\raisebox{.5pt}{\textcircled{\raisebox{-.9pt}{#1}}}}
\newcommand{\definition}[1]{\textcolor{Red3}{\textbf{#1}}\index{#1}}
\newcommand{\example}[1]{\textcolor{Green4}{\textbf{#1}}}
\newcommand{\Var}{\mathrm{Var}}
\newcommand{\Bias}{\mathrm{Bias}}
\newcommand{\Cov}{\mathrm{Cov}}
\newcommand{\highspace}{\vspace{1.2em}\noindent}
\newcommand{\tr}{\mathrm{tr}}
\newenvironment{rowequmat}[1]{\left(\array{@{}#1@{}}}{\endarray\right)}
\newenvironment{rowequmatbra}[1]{\left[\array{@{}#1@{}}}{\endarray\right]}

\begin{document}
    \newcounter{definition}[section]
    \newcounter{example}[section]

    \newtcolorbox[use counter = definition]{definitionbox}{%
        colback=red!5!white,
        colframe=red!75!black,
        fonttitle=\bfseries,
        title=Definition \thetcbcounter %
    }

    \newtcolorbox[use counter = example]{examplebox}{%
        breakable,
        enhanced,
        colback=Green4!5!white,
        colframe=Green4!75!black,
        fonttitle=\bfseries,
        title=Example \thetcbcounter %
    }

    \author{260236}
	\title{Code Transformation and Optimization - Notes}
	\date{\printdayoff\today}
	\maketitle

	\newpage

    \section*{Preface}

    Every theory section in these notes has been taken from two sources:
    \begin{itemize}
        \item None
    \end{itemize}
    About:
    \begin{itemize}
        \item[\faIcon{github}] \href{https://github.com/AndreVale69/HPC-E-PoliMI-university-notes}{GitHub repository}
    \end{itemize}
    
    \newpage
	
	\tableofcontents
	
	\newpage

    \section{Introduction}

    \subsection{Why, When, What, Where?}

    The introduction to this course begins with some basic questions.

    \begin{flushleft}
        \large
        \textcolor{Red3}{\textbf{Why compiling?}}
    \end{flushleft}
    Although it's a trivial question, the reason is not so clear. You could choose interpreters in spite of compilers. Well, actually the \textbf{goal of an \definition{interpreter} is to read one statement/line of code at a time and perform the specified actions}.

    \begin{flushleft}
        \textcolor{Red2}{\faIcon{exclamation-triangle} \textbf{Disadvantage}}
    \end{flushleft}
    There is a \textbf{short delay before execution starts}.

    \begin{flushleft}
        \textcolor{Green3}{\faIcon{check} \textbf{Advantages}}
    \end{flushleft}
    We can do \textbf{interactive execution} (in other words, we can execute partially written code). Also, the interpreter \textbf{avoids compilation overhead if the code is not executed}.

    \highspace
    In contrast, the \textbf{goal of a \definition{compiler} is to translate a compilation unit into machine code}.

    \begin{flushleft}
        \textcolor{Green3}{\faIcon{check} \textbf{Advantages}}
    \end{flushleft}
    \textbf{Optimizes across different statements}; \textbf{faster execution} once the code is compiled; finally, the \textbf{compilation needs to be performed only once}.

    \newpage

    \bibliography{bibtex}{}
    \bibliographystyle{plain}

    \newpage

    \printindex
\end{document}