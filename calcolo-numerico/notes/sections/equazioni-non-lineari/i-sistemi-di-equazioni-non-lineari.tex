\subsection{I sistemi di equazioni non lineari}

Di solito i metodi presentati nelle pagine precedenti vengono inseriti in dei sistemi. Nella realtà ci sono varie condizioni che influiscono sul sistema in analisi. È per questo motivo che si introducono i sistemi.

\highspace
Si consideri un generale sistema di equazioni non lineari:
\begin{equation*}
    \begin{cases}
        f_{1}\left(x_{1}, x_{2}, \dots, x_{n}\right) = 0 \\
        f_{2}\left(x_{1}, x_{2}, \dots, x_{n}\right) = 0 \\
        \vdots \\
        f_{n}\left(x_{1}, x_{2}, \dots, x_{n}\right) = 0 \\
    \end{cases}
\end{equation*}
Dove $f_{1}, \dots, f_{n}$ sono funzioni non lineari. Si pongono i seguenti vettori:
\begin{itemize}
    \item $\mathbf{f} \equiv \left(f_{1}, \dots, f_{n}\right)^{T}$
    \item $\mathbf{x} \equiv \left(x_{1}, \dots, x_{n}\right)^{T}$
\end{itemize}
Con l'obbiettivo di riscrivere il sistema in maniera più agevole:
\begin{equation*}
    \mathbf{f}\left(\mathbf{x}\right) = \mathbf{0}
\end{equation*}
\begin{examplebox}[: esempio di sistema non lineare]
    Un esempio banale di sistema non lineare:
    \begin{equation*}
        \begin{cases}
            f_{1}\left(x_{1}, x_{2}\right) = x_{1}^{2} + x_{2}^{2} - 1 = 0 \\
            \\
            f_{2}\left(x_{1}, x_{2}\right) = \sin\left(\pi \dfrac{x_{1}}{2}\right) + x_{2}^{3} = 0 \\
        \end{cases}
    \end{equation*}
\end{examplebox}

\noindent
Prima di estendere i metodi di Newton e delle secanti si introduce la matrice Jacobiana. 

\begin{definitionbox}[: matrice Jacobiana]
    Senza entrare troppo nel gergo matematico (non è l'obbiettivo del corso), la \definition{matrice Jacobiana} di una funzione è quella \textbf{matrice i cui elementi sono le derivate parziali prime della funzione}.
    \begin{equation}\label{eq: matrice Jacobiana}
        \mathbf{J}_{\mathbf{f}} = 
        \begin{rowequmat}{c c c}
            \dfrac{\partial f_{1}}{\partial x_{1}} & \cdots & \dfrac{\partial f_{1}}{\partial x_{n}} \\ [.3em]
            \vdots & \ddots & \vdots \\ [.3em]
            \dfrac{\partial f_{m}}{\partial x_{1}} & \cdots & \dfrac{\partial f_{m}}{\partial x_{n}}
        \end{rowequmat}
    \end{equation}
    Che può essere riscritto in modo più leggibile come:
    \begin{equation}\label{eq: matrice Jacobiana scritta come funzione vettoriale}
        \left(\mathbf{J}_{\mathbf{f}}\right)_{ij} \equiv \dfrac{\partial f_{i}}{\partial x_{j}} \hspace{2em} i,j = 1, \dots, n
    \end{equation}
    Dove rappresenta la derivata parziale della funzione $f_{i}$ rispetto a $x_{j}$.
\end{definitionbox}

\noindent
Il metodo di Newton e delle secanti può essere esteso sfruttando la matrice Jacobiana:
\begin{itemize}
    \item Il \textbf{metodo di Newton} usando un sistema di equazioni non lineari diventa: dato $\mathbf{x}^{\left(0\right)} \in \mathbb{R}^{n}$, per $k = 0, 1, \dots$, fino a convergenza
    \begin{equation}\label{eq: metodo di Newton in un sistema non lineare}
        \begin{array}{l c l}
            \text{risolvere}&& \mathbf{J}_{\mathbf{f}}\left(\mathbf{x}^{\left(k\right)}\right)\boldsymbol{\delta} \mathbf{x}^{\left(k\right)} = -\mathbf{f}\left(\mathbf{x}^{\left(k\right)}\right) \\ [1em]
            %
            \text{porre}    && \mathbf{x}^{\left(k+1\right)} = \mathbf{x}^{\left(k\right)} + \boldsymbol{\delta}\mathbf{x}^{\left(k\right)}
        \end{array}
    \end{equation}
    Se ne deduce che venga richiesto ad ogni passo la soluzione di un sistema lineare di matrice $\mathbf{J}_{\mathbf{f}}\left(\mathbf{x}^{\left(k\right)}\right)$.

    \item Il \textbf{metodo delle secanti} usando un sistema di equazioni non lineari si basa sulla matrice Jacobiana e sul metodo di Broyden.

    L'\textbf{idea di base} è sostituire le matrici Jacobiane $\mathbf{J}_{\mathbf{f}}\left(\mathbf{x}^{\left(k\right)}\right)$ (per $k \ge 0$) con delle matrici chiamate $\mathrm{B}_{k}$, definite ricorsivamente a partire da una matrice $\mathrm{B}_{0}$ che sia una approssimazione di $\mathbf{J}_{\mathbf{f}}\left(\mathbf{x}^{\left(0\right)}\right)$.

    Dato $\mathbf{x}^{\left(0\right)} \in \mathbb{R}^{n}$, data $\mathrm{B}_{0} \in \mathbb{R}^{n \times n}$ per $k = 0, 1, \dots$, fino a convergenza
    \begin{equation}\label{eq: metodo delle secanti in un sistema non lineare}
        \begin{array}{l c l}
            \text{risolvere}&& \mathrm{B}_{k}\boldsymbol{\delta}\mathbf{x}^{\left(k\right)} = -\mathbf{f}\left(\mathbf{x}^{\left(k\right)}\right) \\ [1em]
            %
            \text{porre}    && \mathbf{x}^{\left(k+1\right)} = \mathbf{x}^{\left(k\right)} + \boldsymbol{\delta}\mathbf{x}^{\left(k\right)} \\ [1em]
            %
            \text{porre}    && \boldsymbol{\delta}\mathbf{f}^{\left(k\right)} = \mathbf{f}\left(\mathbf{x}^{\left(k+1\right)}\right) - \mathbf{f}\left(\mathbf{x}^{\left(k\right)}\right) \\ [1em]
            %
            \text{calcolare}&& \mathrm{B}_{k+1} = \mathrm{B}_{k} + \dfrac{
                \left(\boldsymbol{\delta}\mathbf{f}^{\left(k\right)} - \mathrm{B}_{k}\boldsymbol{\delta}\mathbf{x}^{\left(k\right)}\right)\boldsymbol{\delta}\mathbf{x}^{\left(k\right)^{T}}
            }{
                \boldsymbol{\delta}\mathbf{x}^{\left(k\right)^{T}} \boldsymbol{\delta}\mathbf{x}^{\left(k\right)}
            }
        \end{array}
    \end{equation}
    Da notare che non si chiede alla successione $\left\{\mathrm{B}_{k}\right\}$ così costruita di convergere alla vera matrice Jacobiana $\mathbf{J}_{\mathbf{f}}\left(\boldsymbol{\alpha}\right)$ ($\boldsymbol{\alpha}$ è la radice del sistema); questo risultato non è garantito tuttavia.
\end{itemize}