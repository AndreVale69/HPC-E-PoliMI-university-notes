\subsection{Testing frameworks}

In software development, we typically use \textbf{unit testing frameworks} such as the xUnit frameworks (for example, JUnit and NUnit), which allow us to run unit tests to determine whether different parts of the code behave as expected under different circumstances.

\highspace
The elements of unit test frameworks:
\begin{itemize}
    \item \textbf{Test Runner}: it's a \emph{component that orchestrates the execution of tests and delivers the result to the user}. The runner can use a graphical interface, a textual interface or return a special value to indicate the results of the execution of the tests.
    
    \item \textbf{Test Case}: in most cases this \emph{is a class} from which our application-specific code inherits.

    \item \textbf{Test Fixture}: it represents the \emph{preparation needed to set up the initial state required for a test case before the test, and to return to the original state after the test}.
    
    \item \textbf{Test Suite}: this is a \emph{collection of test cases that share the same fixture}.
    
    \item \textbf{Assertions}: the \emph{functions/macros that check the state or output of the system under test} (oracles).
\end{itemize}
Some examples of frameworks: for Java are \href{https://en.wikipedia.org/wiki/JUnit}{JUnit} (\href{https://junit.org/junit5/docs/current/user-guide/#writing-tests}{here is an example}) and \href{https://jmeter.apache.org/}{Apache JMeter}, for C++ are \href{https://google.github.io/googletest/}{Google Test} (\href{https://google.github.io/googletest/primer.html#simple-tests}{here} is an example of a factorial function), \href{https://en.wikipedia.org/wiki/CppUnit}{CppUnit}, \href{https://github.com/CxxTest/cxxtest}{CxxTest}, \href{https://learn.microsoft.com/en-us/visualstudio/test/writing-unit-tests-for-c-cpp?view=vs-2022}{Microsoft Unit Testing Framework for C++}.