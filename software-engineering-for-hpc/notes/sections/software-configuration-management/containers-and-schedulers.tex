\subsection{Containers and Schedulers}

In general, software should run correctly on multiple environments, such as a developer's laptop. Unfortunately, there are some problems, such as different versions of libraries and middleware.

\highspace
An optimal solution is containers. A \definition{Container} consists of an \textbf{entire runtime environment}: an application plus all its dependencies, libraries and other binaries and configuration files needed to run it, \textbf{bundled into one package}.

\highspace
A similar concept is the \definition{Virtual Machine}. A Virtual Machine, commonly shortened to just VM, is no different than any other physical computer like a laptop, smart phone, or server. It has a CPU, memory, disks to store your files, and can connect to the internet if needed. While the parts that make up your computer (called hardware) are physical and tangible, VMs are often thought of as virtual computers or software-defined computers within physical servers, existing only as code.\footnote{\href{https://azure.microsoft.com/en-us/resources/cloud-computing-dictionary/what-is-a-virtual-machine/}{What is a virtual machine (VM)?}}

\highspace
The differences between containers and virtual machines are as follows:
\begin{itemize}
    \item \textbf{Containers}
    \begin{itemize}
        \item Rely on the underlying operating system.
        \item A container image is in the order of MBs.
        \item Are available almost instantly.
    \end{itemize}
\end{itemize}
\begin{itemize}
    \item \textbf{VMs}
    \begin{itemize}
        \item Each has its own operating system.
        \item A VM image is in the order of GBs.
        \item Require several minutes to boostrap.
    \end{itemize}
\end{itemize}
The most common container frameworks are Docker, which is widely used in classical software development, Singularity, which is used in HPC software, and \href{https://opencontainers.org/}{OCI} (Open Container Initiative).

\begin{flushleft}
    \textcolor{Green3}{\faIcon{book} \textbf{Terminology}}
\end{flushleft}
The terminology used when talking about containers is as follows:
\begin{itemize}
    \item \definition{Image}. This is a read-only \textbf{template that contains instructions for creating a container} (such as Docker). An image can be based on another image, with some customization.

    \item \definition{Container}. An \textbf{executable instance of an image}. Containers can be created, started, stopped, moved or deleted. They can be connected to one or more networks and have storage attached to them.

    \item \href{https://docs.docker.com/get-started/02_our_app/}{Dockerfile} is an example of one of the frameworks and contains the instructions for creating and running a new image.
\end{itemize}