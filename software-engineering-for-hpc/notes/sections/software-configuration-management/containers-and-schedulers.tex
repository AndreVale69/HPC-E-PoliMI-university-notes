\subsection{Containers and Schedulers}

In general, software should run correctly on multiple environments, such as a developer's laptop. Unfortunately, there are some problems, such as different versions of libraries and middleware.

\highspace
An optimal solution is containers. A \definition{Container} consists of an \textbf{entire runtime environment}: an application plus all its dependencies, libraries and other binaries and configuration files needed to run it, \textbf{bundled into one package}.

\highspace
A similar concept is the \definition{Virtual Machine}. A Virtual Machine, commonly shortened to just VM, is no different than any other physical computer like a laptop, smart phone, or server. It has a CPU, memory, disks to store your files, and can connect to the internet if needed. While the parts that make up your computer (called hardware) are physical and tangible, VMs are often thought of as virtual computers or software-defined computers within physical servers, existing only as code.\footnote{\href{https://azure.microsoft.com/en-us/resources/cloud-computing-dictionary/what-is-a-virtual-machine/}{What is a virtual machine (VM)?}}

\highspace
The differences between containers and virtual machines are as follows:
\begin{itemize}
    \item \textbf{Containers}
    \begin{itemize}
        \item Rely on the underlying operating system.
        \item A container image is in the order of MBs.
        \item Are available almost instantly.
    \end{itemize}
\end{itemize}
\begin{itemize}
    \item \textbf{VMs}
    \begin{itemize}
        \item Each has its own operating system.
        \item A VM image is in the order of GBs.
        \item Require several minutes to boostrap.
    \end{itemize}
\end{itemize}
The most common container frameworks are Docker, which is widely used in classical software development, Singularity, which is used in HPC software, and \href{https://opencontainers.org/}{OCI} (Open Container Initiative).

\begin{flushleft}
    \textcolor{Green3}{\faIcon{book} \textbf{Terminology}}
\end{flushleft}
The terminology used when talking about containers is as follows:
\begin{itemize}
    \item \definition{Image}. This is a read-only \textbf{template that contains instructions for creating a container} (such as Docker). An image can be based on another image, with some customization.

    \item \definition{Container}. An \textbf{executable instance of an image}. Containers can be created, started, stopped, moved or deleted. They can be connected to one or more networks and have storage attached to them.

    \item \href{https://docs.docker.com/get-started/02_our_app/}{Dockerfile} is an example of one of the frameworks and contains the instructions for creating and running a new image.
\end{itemize}

\newpage

\subsubsection{Singularity}

\textbf{Singularity} is designed to \textbf{run complex applications on HPC clusters} in a simple, portable and reproducible way.

\highspace
This framework has the following features:
\begin{itemize}
    \item Verifiable reproducibility and security, using cryptographic signatures, an immutable container image format, and in-memory decryption.

    \item Integration over isolation by default. Easily make use of GPUs, high speed networks, parallel filesystems on a cluster or server by default.

    \item Mobility of compute. The single file SIF container format is easy to transport and share.

    \item A simple, effective security model. You are the same user inside a container as outside, and cannot gain additional privilege on the host system by default.
\end{itemize}
However, the previous features are detailed explained \href{https://docs.sylabs.io/guides/latest/user-guide/introduction.html#why-use-singularityce}{here}.

\longline

\subsubsection{Slurm}

Along with Singularity, Slurm is also commonly used. \textbf{Simple Linux Utility for Resource Management} (SLURM) is a free and open source \textbf{job scheduler} (see definition below) for Linux and Unix-like kernels, used by many of the world's supercomputers and computer clusters.

\highspace
Slurm is the workload manager on approximately 60\% of the TOP500 supercomputers.

\highspace
A \definition{Resource Manager} \textbf{co-ordinates the actions of all other components in the batch system} by maintaining a database of all resources, submitted requests and running jobs.

\highspace
In addition, a \definition{Job Scheduler} takes the node and job information from the resource manager and \textbf{creates a list}, sorted by job priority, \textbf{that tells the resource manager when and where to run each job}.

\highspace
Slurm was \textbf{originally} designed as a \textbf{simple resource manager}, capable only of allocating whole nodes to jobs. It has \textbf{evolved} into a \textbf{comprehensive workload scheduler capable of managing the most demanding workflows} on many of the world's largest computers. Its design goals are to be open source, portable, scalable, fault tolerant, secure and sys admin friendly.

\newpage

\begin{flushleft}
    \textcolor{Green3}{\faIcon{book} \textbf{Terminology}}
\end{flushleft}
Some of the main concepts of Slurm:
\begin{itemize}
    \item \textbf{Job}: resource allocation request, two types:
    \begin{enumerate}
        \item Batch script to be queued for later execution.
        \item Interactive job for which the user awaits resource allocation and then makes real-time use of it.
    \end{enumerate}

    \item \textbf{Job Step}: a set of parallel tasks, typically an MPI application.

    \item \textbf{Node}: computational element.
    
    \item \textbf{Cluster}: a collection of nodes sharing the same network.
    
    \item \textbf{Federation}: a collection of clusters sharing a standard configuration DB.
    
    \item \textbf{Job Partition}: It is a job queue. Note that a job can be partitioned into multiple parts.
    
    \item \textbf{Account}: an access group that gives users access to resources in a cluster.
    
    \item \textbf{Slurm Association}: a combination of the cluster, account, username, and (optional) partition name.
\end{itemize}
\begin{flushleft}
    \textcolor{Green3}{\faIcon{atlas} \textbf{Commands}}
\end{flushleft}
And here is a list of the most common Slurm commands (\href{https://slurm.schedmd.com/pdfs/summary.pdf}{summary of commands}):
\begin{itemize}
    \item \href{https://slurm.schedmd.com/slurmd.html}{\texttt{slurmd}}: controls the execution of jobs and job steps on a node

    \item \href{https://slurm.schedmd.com/slurmctld.html}{\texttt{slurmctld}}: monitors the status of resources, decides when and where to initiate jobs and job steps, and processes almost all user commands.

    \item \href{https://slurm.schedmd.com/slurmdbd.html}{\texttt{slurmdbd}}: interfaces with a DBMS such as MariaDB or MySQL, keeps accounting information and manages some configuration information centrally (many limits, fair share information, QOS, licences, etc.).

    \item \href{https://slurm.schedmd.com/slurmrestd.html}{\texttt{slurmrestd}}: interfaces to Slurm via REST.

    \item \href{https://slurm.schedmd.com/sbatch.html}{\texttt{sbatch}}: submits a job script for later execution.

    \item \href{https://slurm.schedmd.com/srun.html}{\texttt{srun}}: submits a job for execution or initiate job steps in real time.

    \item \href{https://slurm.schedmd.com/salloc.html}{\texttt{salloc}}: allocates resources for a job in real time.

    \item \href{https://slurm.schedmd.com/squeue.html}{\texttt{squeue}}: reports the state of jobs or job steps.

    \item \href{https://slurm.schedmd.com/sinfo.html}{\texttt{sinfo}}: reports the state of partitions and nodes managed by Slurm.
\end{itemize}
The default scheduling approach is FIFO. But with the Multifactor Priority plugin it is ordered by priority. There is a detailed explanation in the following \href{https://mmesiti.github.io/slurmprio200428/}{link} (these are slides).