\subsection{Eliciting requirements}

The \definition{Requirements Elicitation} is the \textbf{practice of researching and discovering the requirements of a system from users, customers, and other stakeholders}. The \textbf{goal} of requirements elicitation is to ensure that the software development process is based on a clear and comprehensive understanding of the customer's needs and requirements. To do that, exist a simple and effective tool called \definition{scenario}s.

\begin{definitionbox}
    A \definition{scenario} is a concrete, focused, informal description of a single feature of the system to be.
\end{definitionbox}

\begin{examplebox}[: warehouse on fire]\label{example: warehouse on fire}
    \emph{Bob driving down main street in his patrol car notices smoke coming out of a warehouse. His partner, Alice, reports the emergency from her car.}

    \highspace
    \emph{Alice enters the address of the building, a brief description of its location (i.e. north west corner), and an emergency level. In addition to a fire unit, she requests several paramedic units on the scene given that area appears to be relatively busy. She confirms her input and waits for an acknowledgment.}

    \highspace
    \emph{John, the Dispatcher, is alerted to the emergency by a beep of his workstation. He reviews the information submitted by Alice and acknowledges the report. He allocates a fire unit and two paramedic units to the incident site and sends their estimated time of arrival (ETA) to Alice.}

    \highspace
    \emph{Alice received the acknowledgment and the ETA.}
\end{examplebox}
There are heuristics for finding scenarios, such as asking the customer a few questions:
\begin{itemize}
    \item Which user groups are supported by the system to perform their work?
    \item What are the primary tasks that the system needs to perform?
    \item What data will the actor create, store, change, remove or add in the system?
    \item What external changes does the system need to know about?
    \item What changes or events will the actor of the system need to be informed about?
\end{itemize}
However, it's very important \underline{not} \textbf{to rely on questionnaires alone}! \textbf{Insist on task observation} (if possible), ask to \textbf{speak to the end user}, not just the software contractor, and expect resistance, but try to overcome it.

\newpage

\noindent
Scenarios provide a nice summary of what the requirements analysis team can derive from observation, interviews, analysis of documentation. By generalizing the scenarios, we can get \definition{Use Cases}.

\highspace
To specify a use case, it's very important to follow the following scheme.
\begin{definitionbox}[: Use Cases Schema]
    \begin{itemize}
        \item \textbf{Name of Use Case}
        
        \item \textbf{Actors}
        \begin{itemize}
            \item \emph{Description of Actors involved in use case}.
        \end{itemize}
        
        \item \textbf{Entry condition}
        \begin{itemize}
            \item \dquotes{\emph{When this use case starts the following condition is true...}}.
        \end{itemize}
        
        \item \textbf{Flow of Events}
        \begin{itemize}
            \item \emph{Free form, informal natural language}.
        \end{itemize}
        
        \item \textbf{Exit condition}
        \begin{itemize}
            \item \dquotes{\emph{This use case terminates when the following condition holds...}}.
        \end{itemize}
        
        \item \textbf{Exceptions}
        \begin{itemize}
            \item \emph{Describe what happens if things go wrong}.
        \end{itemize}
        
        \item \textbf{Special Requirements}
        \begin{itemize}
            \item \emph{Nonfunctional Requirements, Constraints}.
        \end{itemize}
    \end{itemize}
\end{definitionbox}

\noindent
The following \textbf{suggestions} are useful in defining an appropriate use case:
\begin{itemize}
    \item Use cases named with verbs that indicate what the user is trying to accomplish
    \item Actors named with nouns
    \item Use cases steps in active voice
    \item The causal relationship between steps should be clear
    \item A use case per user transaction
    \item Separate description of exceptions
    \item Keep use cases small (no more than two/three pages)
    \item The steps accomplished by actors and those accomplished by the system should be clearly distinguished (this helps us in identifying the boundaries of the system)
\end{itemize}

\newpage

\noindent
First of all, we present an example of a \textbf{bad use case}.
\begin{examplebox}[: bad use case]
    \example{Example} of a bad use case referring to the ambulance dispatching example on page~\pageref{example: ambulance dispatching system}:
    \begin{itemize}
        \item Use case name: Accident
        \item Participating Actors:
        \begin{itemize}
            \item Field Officer
        \end{itemize}
        \item Flow of Events:
        \begin{enumerate}
            \item The Field Officer reports the accident
            \item An ambulance is dispatched
            \item The Dispatcher is notified when the ambulance arrives on site
        \end{enumerate}
    \end{itemize}
    The \underline{errors} are as follows:
    \begin{itemize}
        \item In the \emph{use case name} field, the \textbf{word is a noun}. It's better to use a verb that indicates what the user is trying to achieve.
        
        \item The Dispatcher actor is \textbf{not declared} in the \emph{Participating Actors} field, but is mentioned in the \emph{Flow of Events} field.

        \item There are two main errors in the \emph{Flow of Events} section: the first is in the sentence \dquotes{\emph{An ambulance is dispatched}}. But \textbf{who sends it?} The second is in the third sentence, because \textbf{who notifies the Dispatcher?}
    \end{itemize}
\end{examplebox}

\noindent
Now we present an example of a \emph{well composed} use case.
\begin{examplebox}[: use case \texttt{ReportEmergency} with reference to the example on page~\pageref{example: warehouse on fire}]
    There are two \textbf{actors} involved:
    \begin{itemize}
        \item Field Officer (Bob and Alice in the Scenario)
        \item Dispatcher (John in the Scenario)
    \end{itemize}
    The \textbf{Entry Condition} is always true because an emergency can be reported at any time. The \textbf{sequence of events} is as follows:
    \begin{itemize}
        \item The \textbf{FieldOfficer} activates the Report Emergency function of her terminal.

        \item \textbf{Friend} (the system to be developed) responds by presenting a form to the officer.
        
        \item The FieldOfficer fills the form, by selecting the emergency level, type, location, and brief description of the situation. The FieldOfficer also describes possible responses to the emergency situation. Once the form is completed, the FieldOfficer submits the form.
        
        \item At which point, the \textbf{Dispatcher} is notified.
        
        \item The Dispatcher reviews the submitted information and allocates resources by invoking the AllocateResources use case. The Dispatcher selects a response and acknowledges the emergency report.
    \end{itemize}
    The \textbf{Exit Condition} is the following: the FieldOfficer has received the acknowledgment and the selected response.

    \highspace
    There are two possible \textbf{exceptions}:
    \begin{itemize}
        \item The FieldOfficer is notified immediately if the connection between her terminal and the control room is lost.
        \item The Dispatcher is notified immediately if the connection between any logged in FieldOfficer and the control room is lost.
    \end{itemize}
    And the \textbf{special requirements} are:
    \begin{itemize}
        \item The FieldOfficer's report is acknowledged within 30 seconds;
        \item The selected response arrives no later than 30 seconds after it is sent by the Dispatcher.
    \end{itemize}
\end{examplebox}

\begin{examplebox}[: use case \texttt{AllocateResources} with reference to the example on page~\pageref{example: warehouse on fire}]
    \begin{itemize}
        \item Use case name: AllocateResources
        \item Participating Actors:
        \begin{itemize}
            \item Dispatcher (John in the Scenario. The Dispatcher allocates a resources to an Emergency if the resource is available; of course, he also updates and removes Emergency Incidents, Actions, and Requests in the system)
            
            \item Resource Allocator (the Resource Allocator is responsible for allocating resources in case they are scarce)
            
            \item Resources (the Resources that are allocated to the Emergency)
        \end{itemize}
        \item Entry Condition:
        \begin{itemize}
            \item An Incident has been opened
        \end{itemize}
        \item Flow of Events:
        \begin{itemize}
            \item The Dispatcher selects the types and number of Resources that are needed for the incident.
            
            \item Friend replies with a list of Resources that fulfill the Dispatcher's request.
            
            \item The Dispatcher selects the Resources from the list and allocates them for the incident.
            
            \item Friend automatically notifies the Resources.

            \item The Resources send a confirmation.
        \end{itemize}
        \item Exit Condition:
        \begin{itemize}
            \item The use case terminates when the resource is committed.
            \item The selected Resource is now unavailable to any other Emergences or Resource Requests.
        \end{itemize}
        \item Exceptions:
        \begin{itemize}
            \item If the list of Resources provided by Friend is insufficient to fulfill the needs of the emergency, the Dispatcher informs the Resource Allocator.
            \item The Resource Allocator analyzes the situation and selects new Resources by decommitting them from their previous work.
            \item Friend automatically notifies the Resources and the Dispatcher.
            \item The Resources send a confirmation.
        \end{itemize}
    \end{itemize}
\end{examplebox}