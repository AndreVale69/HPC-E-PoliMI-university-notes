\section{HPC Software, Relevant Qualities and Systems Engineering Methods}

\subsection{High Performance Computing Software}

There's no single definition of HPC, but it can be explained in a number of ways:
\begin{definitionbox}
    The practice of aggregating computing power in a way that delivers much high performance than one could get out of a typical desktop computer or workstation to solve large problems in science, engineering, or business.

    \highspace
    Thousands of processors working in parallel to analyze billions of pieces of data in real time, performing calculations thousands of times faster than a normal computer.

    \highspace
    The use of parallel processing for running advanced, large-scale application programs efficiently, reliably and very quickly on supercomputer systems.

    \highspace
    The platform technology concerned with programming for performance, where performance takes the broad meaning of:
    \begin{itemize}
        \item Speed (reducing time to solution);
        \item Energy efficiency (doing more with less power);
        \item Upscaling (handling larger problems such as simulating a wing aand then a full plane, or a cell and then an organ);
        \item High throughput (the ability to handle large volumes of data in near real-time, as required in the financial services industry, telecoms or satellite imagery).
    \end{itemize}
\end{definitionbox}

\noindent
As \definition{Parallel and Distributed Computing (PDC)} exist, it is necessary to explain the difference. The main characteristics of PDC are:
\begin{itemize}
    \item \textbf{Concurrency}: it is a property of software. \textbf{A piece of software is also concurrent if it can have more than one active execution context}.

    \item \textbf{Parallelism}: it is a property of software. \textbf{The execution of different tasks/pieces of software at the same time}.

    \item \textbf{Distribution}. \textbf{The execution of different tasks/pieces of software on physically distinct computing nodes connected through a network, lack of a global clock}.
\end{itemize}
\textbf{PDCs} are \textbf{multi-core machines}, whereas \textbf{HPCs} are \textbf{quantum computers}. However, \textbf{both share parallel machines}, \textbf{HPC clusters} and \textbf{cloud infrastructures}.

\newpage

\noindent
There are \textbf{two categories} of HPC software:
\begin{itemize}
    \item \definition{Compute-intensive applications}. These are \textbf{complex calculations that require a large number of computing resources}. They also often require parallel computing.

    \item \definition{Data-intensive applications}. They \textbf{focus on processing, storing and retrieving large amounts of data}. Typically built as distributed systems to ensure reliability and scalability.
\end{itemize}