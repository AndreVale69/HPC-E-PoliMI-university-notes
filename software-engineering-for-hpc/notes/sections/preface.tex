\section*{Preface}

Each theory section in these notes has been taken from the following sources:
\begin{itemize}
    \item Course slides.\cite{slides}
    \item SE4HPC Exercise Book.\cite{exerciseBook}
\end{itemize}
About:
\begin{itemize}
    \item[\faIcon{github}] \href{https://github.com/PoliMI-HPC-E-notes-projects-AndreVale69/HPC-E-PoliMI-university-notes}{GitHub repository}
\end{itemize}
These notes are an unofficial resource and shouldn't replace the course material or any other book on software engineering. It is not made for commercial purposes. I've made the following notes to help me improve my knowledge and maybe it can be helpful for everyone.

As I have highlighted, a student should choose the teacher's material or a book on the topic. These notes can only be a helpful material.

\highspace
During the Software Engineering for HPC course, we created two projects:
\begin{enumerate}
    \item Requirement Engineering and Design Project (section 1 to 5). More information in the following repository (actually private)
    \begin{itemize}
        \item \faIcon{github} \href{https://github.com/PoliMI-HPC-E-notes-projects-AndreVale69/SE4HPC_project_RD}{SE4HPC\_project\_RD}
    \end{itemize}
    
    \item DevOps Project (section 6 to 7). More information in the following repositories
    \begin{itemize}
        \item \faIcon{github} \href{https://github.com/AndreVale69/SE4HPC_project_part1}{SE4HPC\_project\_part1}

        \item \faIcon{github} \href{https://github.com/AndreVale69/SE4HPC_project_part2}{SE4HPC\_project\_part2}
    \end{itemize}
\end{enumerate}