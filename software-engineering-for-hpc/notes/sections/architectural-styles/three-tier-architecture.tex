\subsection{Three-Tier Architecture}

The following is a summary of the \href{https://www.ibm.com/topics/three-tier-architecture}{IBM guide}.

\highspace
\definition{Three-tier architecture} is a well-established software application architecture that \textbf{organizes applications into three logical and physical computing tiers}: 
\begin{itemize}
    \item The \textbf{presentation} tier, or user interface;
    
    \item The \textbf{application} tier, where data is processed;
    
    \item The \textbf{data} tier, where application data is stored and managed.
\end{itemize}

\begin{flushleft}
    \textcolor{Green3}{\textbf{\faIcon{check} Benefits}}
\end{flushleft}
The chief benefit of three-tier architecture is its \textbf{logical and physical separation} of functionality. Each tier can run on a separate operating system and server platform - for example, web server, application server, database server - that best fits its functional requirements. And each tier runs on at least one dedicated server hardware or virtual server, so the services of \textbf{each tier can be customized and optimized without impacting the other tiers}. Other benefits include:
\begin{itemize}
    \item \textbf{Faster development}: Because \emph{each tier can be developed simultaneously by different teams}, an organization can bring the application to market faster. And programmers can use the latest and best languages and tools for each tier.
    
    \item \textbf{Improved scalability}: \emph{Any tier can be scaled independently} of the others as needed.

    \item \textbf{Improved reliability}: An outage in one tier is less likely to impact the availability or performance of the other tiers.

    \item \textbf{Improved security}: Because the \emph{presentation tier and data tier can't communicate directly}, a well-designed application tier can function as an internal firewall, preventing SQL injections and other malicious exploits.
\end{itemize}

\subsubsection{N-tier architecture}

\definition{N-tier architecture} (also called or multitier architecture) refers to any application architecture with \textbf{more than one tier}. But applications with more than three layers are \underline{rare} because extra layers offer \textbf{few benefits} and can make the \textbf{application slower}, \textbf{harder to manage} and \textbf{more expensive to run}. As a result, n-tier architecture and multitier architecture are usually synonyms for three-tier architecture.