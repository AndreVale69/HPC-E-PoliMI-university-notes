\subsection{Data-Intensive applications}

Before we introduce the architectural styles for data-intensive applications, we explain the difference between batch and stream processing.

\highspace
\definition{Batch processing} is a method of running software programs called jobs in batches automatically. While users are required to submit the jobs, no other interaction by the user is required to process the batch.

\highspace
\definition{Stream processing} (also known as event stream processing, data stream processing, or distributed stream processing) is a programming paradigm which views streams, or sequences of events in time, as the central input and output objects of computation.

\begin{table}[!htp]
    \centering
    \begin{tabular}{@{} p{16em} p{16em} @{}}
        \toprule
        \textbf{Batch} & \textbf{Stream} \\
        \midrule
        Has access to all data. & Computes a function of one data element, or a smallish window of recent data. \\
        \cmidrule{1-2}
        Might compute something big and complex. & Computes something relatively simple. \\
        \cmidrule{1-2}
        Is generally more concerned with throughput than latency of individual components of the computation. & Needs to complete each computation in near-real-time - probably seconds at most. \\
        \cmidrule{1-2}
        Has latency measured in minutes or more. & Computations are generally independent. \\
        \cmidrule{1-2}
        & Asynchronous - source of data doesn't interact with the stream processing directly, like by waiting for an answer. \\
        \bottomrule
    \end{tabular}
    \caption{Batch vs Stream processing.}
\end{table}