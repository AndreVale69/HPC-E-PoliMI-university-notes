\subsection{Architecture and multiple structures}

There is a set of structures relevant to the software:
\begin{itemize}
    \item \definition{Component-and-connector (C\&C)} structures. Describe how the system is structured as a \textbf{set of elements} that have \textbf{runtime behavior} (called components) and interactions (called connectors).
    \begin{itemize}
        \item The \textbf{components} are the principal units of computation (for \example{example} the clients, servers, services, etc.)

        \item The \textbf{connectors} represent communication (for \example{example} request-response mechanisms, pipes, asynchronous messages, etc.)
    \end{itemize}
    The \emph{\textbf{\underline{purpose}}} of these structures is to \textbf{enable us to answer questions} such as:
    \begin{itemize}
        \item \emph{What are the major executing components and how do they interact at runtime?}

        \item \emph{What are the major shared data stores?}
        
        \item \emph{Which parts of the system are replicated?}

        \item \emph{How does data progress through the system?}

        \item \emph{Which parts of the system can run in parallel?}

        \item \emph{How does the system's structure evolve during execution?}
    \end{itemize}
    Also, \textbf{allow us to study runtime properties} such as availability and performance.

    \item \definition{Module} structures. Show how a system is structured as \textbf{a set of code or data units} that have to be procured or constructed, \textbf{together with their relations}. An \example{example} of modules: packages, classes, functions, libraries, layers, database tables, etc.
    
    Modules constitute \textbf{implementation units} that can be used as the basis for work splitting (identifying functional areas of responsibility). Typical relations among modules are: uses, is-a (generalization), is-part-of.

    The \emph{\textbf{\underline{purpose}}} of these structures is to \textbf{allow us to answer questions} such as:
    \begin{itemize}
        \item \emph{What is the primary functional responsibility assigned to each module?}

        \item \emph{What other software elements is a module allowed to use?}

        \item \emph{What other software does it actually use and depend on?}

        \item \emph{What modules are related to other modules by generalization or specialization (i.e. inheritance) relationships?}
    \end{itemize}
    Can also be used to answer questions about the \textbf{impact on the system when the responsibilities assigned to each module change}.

    \item \definition{Allocation} structures. Define \textbf{how the elements} from component-and-connector or module structures \textbf{map} onto things that are not software. For \example{example} hardware (possibly virtualized), file systems, teams. Some typical allocation structures include deployment structure, implementation structure, work assignment structure.
\end{itemize}