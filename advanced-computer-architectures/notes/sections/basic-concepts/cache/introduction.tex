\subsection{Cache}

\subsubsection{Introduction}

The cache is introduced to increase the performance of a computer through the memory system in order to:
\begin{itemize}
    \item Provide the user the illusion to use a memory that is simultaneously large and fast.
    \item Provide the data to the processor at high frequency.
\end{itemize}
It takes advantage from the \indexdefinition{Locality of Reference}.

\begin{definitionbox}[: Locality of Reference]
    \href{https://www.geeksforgeeks.org/locality-of-reference-and-cache-operation-in-cache-memory/}{Locality of reference} refers to a phenomenon in which a \textbf{computer program tends to access same set of memory locations for a particular time period}. 
    
    In other words, \indexdefinition{Locality of Reference} refers to the tendency of the computer program to access instructions whose addresses are near one another. The property of locality of reference is mainly shown by loops and subroutine calls in a program.
\end{definitionbox}

\noindent
There are two types of Locality of Reference:
\begin{itemize}
    \item \textbf{Temporal Locality}
    \begin{definitionbox}[: Temporal Locality]
        \indexdefinition{Temporal Locality} means that a \textbf{instruction which is recently executed have high chances of execution again}. So the instruction is kept in cache memory such that it can be fetched easily and takes no time in searching for the same instruction.
    \end{definitionbox}

    \item \textbf{Spatial Locality}
    \begin{definitionbox}[: Spatial Locality]
        \indexdefinition{Spatial Locality} means that \textbf{all those instructions which are stored nearby to the recently executed instruction have high chances of execution}. It refers to the use of data elements(instructions) which are relatively close in storage locations.
    \end{definitionbox}
\end{itemize}

\newpage

\begin{table}[!htp]
    \centering
    \begin{tabular}{@{} p{16em} p{16em} @{}}
        \toprule
        \textbf{Spatial Locality} & \textbf{Temporal Locality} \\
        \midrule
        In Spatial Locality, nearby instructions to recently executed instruction are likely to be executed soon.
        & 
        In Temporal Locality, a recently executed instruction is likely to be executed again very soon. \\
        \cmidrule{1-2}
        It refers to the tendency of execution which involve a number of memory locations.
        &
        It refers to the tendency of execution where memory location that have been used recently have a access. \\
        \cmidrule{1-2}
        It is also known as locality in space.
        &
        It is also known as locality in time. \\
        \cmidrule{1-2}
        It only refers to data item which are closed together in memory.
        &
        It repeatedly refers to same data in short time span. \\
        \cmidrule{1-2}
        Each time new data comes into execution.
        &
        Each time same useful data comes into execution. \\
        \cmidrule{1-2}
        \example{Example}: Data elements accessed in array (where each time different, or just next, element is being accessing).
        &
        \example{Example}: Data elements accessed in loops (where same data elements are accessed multiple times). \\
        \bottomrule
    \end{tabular}
    \caption{\href{https://www.geeksforgeeks.org/difference-between-spatial-locality-and-temporal-locality/}{Difference between Spatial Locality and Temporal Locality}.}
\end{table}

\begin{flushleft}
    \textcolor{Green3}{\textbf{\faIcon{question-circle} Where can we find the cache?}}
\end{flushleft}
In general, the memory hierarchy is composed of several level. Let us consider 2 levels: cache and main memory. The \textbf{cache} (upper level) is \textbf{smaller}, \textbf{faster} and \textbf{more expensive} than the main memory (lower level). 

\highspace
The \textbf{minimum chunk of data that can be copied in the cache} is the \textbf{\emph{block}} or \textbf{\emph{cache line}}. To exploit the spatial locality, the block size must be a multiple of the word size in memory. So, for example a 128-bit block size is equal to 4 words of 32-bit.

\highspace
The \textbf{number of blocks in cache} is given by:
\begin{equation*}
    \texttt{Number of cache blocks} = \dfrac{\texttt{Cache Size}}{\texttt{Block Size}}
\end{equation*}
For example, if the cache size is 64K-Byte and the block size is 128-bit (16-Byte), then the number of cache blocks is 4K blocks.