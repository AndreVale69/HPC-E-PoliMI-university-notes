\subsection{Exceptions handling}

\subsubsection{Introduction}

We use the term \indexdefinition{exception} to cover not only \textbf{exceptions} but also \textbf{interrupts} and \textbf{faults}. More in general, we consider the following \emph{type of events}:
\begin{itemize}
    \item I/O device request
    \item Invoking OS system call from a user program
    \item Tracing instruction execution
    \item Integer arithmetic overflow/underflow
    \item Floating point arithmetic anomaly
    \item Page fault
    \item Misaligned memory access
    \item Memory protection violation
    \item Hardware/Power failure
\end{itemize}

\begin{flushleft}
    \textcolor{Red3}{\faIcon{exclamation-triangle} \textbf{Causes of Interrupts/Exceptions}}
\end{flushleft}
An \indexdefinition{interrupt} is an \textbf{event that requires the processor's attention}. The causes of an interrupt or exception can be of two types:
\begin{itemize}
    \item \indexdefinition{Asynchronous Exceptions}, when a request comes from an \textbf{external event}, such as:
    \begin{itemize}
        \item \textbf{I/O device service-request}
        \item \textbf{Timer expiration}
        \item \textbf{Power disruptions, hardware failure}
    \end{itemize}
    These events are caused by \textbf{devices external to the CPU and memory} and can be handled after the completion of the current instruction (easier to handle)

    \item \indexdefinition{Synchronous Exceptions}, when a request comes from an \textbf{internal event} (a.k.a. exceptions), such as:
    \begin{itemize}
        \item \textbf{Undefined opcode, privileged instruction}
        \item \textbf{Integer arithmetic overflow, FPU exception}
        \item \textbf{Misaligned memory access}
        \item \textbf{Virtual memory exceptions}: page faults, TLB misses, protection violations
        \item \textbf{Traps}: system calls, e.g., jumps into kernel
    \end{itemize}
\end{itemize}

\newpage

\begin{flushleft}
    \textcolor{Green4}{\faIcon{list-alt} \textbf{Classes of Exceptions}}
\end{flushleft}
Some exceptions are predictable and easier to handle, such as user-requested exceptions. But, some exceptions are unpredictable, such as \dquotes{coerced} exceptions. Other classes of exceptions are:
\begin{itemize}
    \item \indexdefinition{User Requested \dquotes{Exceptions}}, such as I/O events, are \textbf{predictable}. They are treated as exceptions because they use the same mechanisms that are used to save and restore the state; handled after the instruction has completed (\emph{easier to handle}).
    

    \item \indexdefinition{Coerced \dquotes{exceptions}} are caused by some hardware event not under control of the user program; hard to implement because they are \textbf{unpredictable}.
    

    \item \textbf{Masking}. To mask an interrupt is to \textbf{disable} it, so it is deferred or ignored by the processor, while to unmask an interrupt is to enable it.
    
    \indexdefinition{User maskable} interrupts are signals affected by the mask. So, when the interrupt is disabled, the associated interrupt signal may be ignored by the processor, or it may remain pending.

    \indexdefinition{User nonmaskable} interrupts are signals that cannot be disabled and they are not affected by the interrupt mask.


    \item \indexdefinition{Within instructions}. These classes of exceptions are \textbf{synchronous}, since the instruction triggers the exception. 
    
    In this case, the \textbf{instruction must be stopped and restarted}.


    \item \indexdefinition{Between instructions}. These classes of exceptions are asynchronous and they arise from \textbf{catastrophic situations} such as HW malfunctions and \textbf{\underline{always cause program termination}}.
\end{itemize}
Finally, there are two explanations of the terms we will use:
\begin{itemize}
    \item The term \indexdefinition{terminating event} means that the \textbf{program execution always stops after the interrupt/exception}.
    
    \item The term \indexdefinition{resume event} means that the \textbf{program execution continues after the interrupt/exception}.
\end{itemize}